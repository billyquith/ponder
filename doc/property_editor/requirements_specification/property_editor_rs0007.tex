\documentclass[a4paper, twoside]{report}

%-------------------------------------

% Document type
\usepackage{tgsreqspec}
%\usepacakge(tgstechspec}
%\usepackage{tgstechnote}
%\usepackage{tgsproc}
%\usepackage{tgsguidelines}
%\usepackage{tgstestplan}

% Document information
\documentreference{RS-0007}
\projectcode{DMAC}
\documentmajorversion{1}
\documentminorversion{A}
\clientcode{TGS}
\documentauthor{\persons{DEL}{}}
\documentcontroller{\persons{RAC}{}}
\documentquality{\persons{DOC}{}}
\title{CAMP Property Editor}
\date{\today}

% Contractual document ?
%\contractualdocument
\notcontractualdocument

% Document status
\draftdocument
%\underapprovaldocument
%\publisheddocument
%\obsoletedocument

% Document classification
%\publicdocument
%\restricteddocument
\confidentialdocument

% Generate PDF information
\makepdfinfo

%-------------------------------------

\begin{document}

\maketitle

\chapter*{About this document}

\section*{Document history}

% To add a history entry use:
% \addhistory{version}{data}{author}{changes}
% Newer history entries first
\begin{historytable}
\end{historytable}

\section*{Related documents}

% To add a related document entry use:
% \adddocument{document}{version}{parts}{knowledge}
% document: the document reference, title, url...
% version: version of the document if any
% parts: parts of the document which is relevant (pages, sections...)
% knowledge: should be either "Required", or "Recommanded", or "Advised"
\begin{documenttable}
    \adddocument{CAMP RS-0001}{1.A}{}{Required}
\end{documenttable}

\tableofcontents

%-------------------------------------

\chapter{Introduction\label{sec:introduction}}

The CAMP Property Editor is a graphical component which allows to view and edit CAMP properties. The
component is designed to be embedded into an application which allow easy interaction with CAMP
objects.

The property editor is extendable to allow custom display and editing according to CAMP property
types.

This document specifies the requirements for the CAMP Property Editor.

\chapter{Description of needs\label{sec:needs}}

Here is an unordered list of needs for the CAMP Property Editor component.

\begin{itemize}
    \item Based on Qt Model/View framework
    \item Allow synchronous viewing and editing of properties
    \item Ease a future implementation of an asynchronous viewing and editing mode
    \item Allow cancellation of an editing operations
    \item Handle clean and dirty states:
    \begin{itemize}
        \item Clean state correspond typically to the state of the object the last time it was saved
        \item Dirty state correspond to an object which has been modified since the last time it was
saved; the modified properties must be emphasized
    \end{itemize}
    \item Handle all CAMP property types
    \item Allow customization/extension of the property name viewer
    \item Allow customization/extension of the property value viewer and editor
    \item Allow to group properties according to the class hierarchy
    \item Allow property filtering by property name and by tag name using regular expressions
    \item Allow to edit the common properties of several objects inheriting a same base class
\end{itemize}

\chapter{Overview\label{sec:overview}}

The CAMP Property Editor shows the CAMP properties of one or several objects in a manner similar to the Qt
designer property editor which does it for Qt properties (fig. \ref{fig:qt-property-editor}).

\image[0.5]{fig:qt-property-editor}{Qt Designer Property Editor}{images/qt_property_editor}

The editor widget must be composed of two main parts:
\begin{itemize}
    \item A tree like structure in a first column which usually displays the property names. It is
also possible to delegate the property name display to have specific rendering (e.g. for arrays, it
allows to iterate over elements one by one).
    \item A second column allows to view and edit the value associated with the properties:
    \begin{itemize}
        \item If the property is readable, its value is displayed according to its type. Else, a
notification of this \emph{not readable} state is displayed.
        \item If the property is writable, a value editor is displayed when the user click on the
cell. The editor depends of the property types. If the property is not writable, the property value
is grayed and no editor is displayed.
    \end{itemize}
\end{itemize}

The section \ref{sec:property_types} details the different viewers and editors for all supported
CAMP property types.

\section{Filtering and grouping}

The property editor must display all visible properties of a CAMP object. Moreover, it must be possible
to filter the properties according to their names or their tag names using regular expressions.

The property editor must allow to group the properties of a CAMP object according to the class
hierarchy of the object. The figures \ref{fig:without-classes} and \ref{fig:with-classes} show an
example of a CAMP object being displayed by the CAMP property editor with and without grouping. In
the first case, all no classes informations occurs. Each sub-tree directly lists all the properties of the
corresponding CAMP object. In the second case, all the properties are placed under the meta-class
name in which they have been declared. This allows to group properties from the most general sub-set to the
most specific one.

\image[0.5]{fig:without-classes}{Simple property tree}{images/without_classes}
\image[0.5]{fig:with-classes}{Property tree with classes hierarchy}{images/with_classes}

\section{Editing properties}

When a property is beeing edited, it must be possible to cancel the operation at any time. The
previous value is then restored.

The CAMP property editor must allow to identify what properties have been changed from the previous
\emph{clean state}. The clean state corresponds usually to the last time the object has been saved.
All the properties which differ from that state must be emphasized. This way, the user can easily
see what has been modified.

The editor must be able to edit several CAMP objects at the same time. In this case, the CAMP
property editor must list all the of their first common meta-class ancestor. Thus, it the objects
do not inherit a same base class, no property will be listed.

\chapter{Handling Property Types\label{sec:property_types}}

This section will define how the different CAMP property types will be handled by default. This
includes how is represented the property name, the property value and how the property value is
edited. First, we will begin by the default handling if the property type is not handled yet,
or if it does not provide it own display. Then, for each property type, any special handling is
listed.

\section{Default}

This section define how properties will be displayed by default (i.e. when no overriding occurs).

The property name is displayed as a simple text. If the property value is convertible into a string,
this string representation is used as value. Otherwise, the value \emph{not available} is displayed
in bold and italic. There is no editor available.

\section{Boolean}

This section define how boolean properties will be displayed.

The property value is displayed and edited with a check box.

\section{Integer}

The value editor is a spin box for integer number.

\section{Real}

The value editor is a spin box for real number.

\section{String}

The editor is a line edit.

\section{Array}

An array can be displayed in two ways:
\begin{itemize}
    \item In the first column of the property editor, the property name is replaced by the property
name plus an additional spin box. The spin box is used to iterate over the different values of the
array. The value of the property is displayed in the second column according to its type. If the
value as property too, they are listed into a sub-tree as usual.
    \item The wall content of the array is listed in a sub-tree. The first column shows the index of
the value displayed in the second column.
\end{itemize}

The default behavior is the first one because it is more compact and it avoid to query all the array
values.

TODO dynamic array (+,-) and custom dialog

\section{Enumeration}
\section{Object}

\end{document}

